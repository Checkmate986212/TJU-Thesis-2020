% !Mode:: "TeX:UTF-8"
%  Authors: 张井   Jing Zhang: prayever@gmail.com     天津大学2010级管理与经济学部信息管理与信息系统专业硕士生
%           余蓝涛 Lantao Yu: lantaoyu1991@gmail.com  天津大学2008级精密仪器与光电子工程学院测控技术与仪器专业本科生

%%%%%%%%%%%%%%%%% Fonts Definition and Basics %%%%%%%%%%%%%%%%%
\setCJKfamilyfont{黑体}{SimHei} %{Noto Sans CJK SC} %{SimHei}
\setCJKfamilyfont{粗黑体}[Path=fonts/]{SourceHanSansSC-Medium.otf}%{SimHei} %{Noto Sans CJK SC} %{SimHei}
\setCJKfamilyfont{宋体}{SimSun}
\setCJKfamilyfont{粗宋体}[Path=fonts/]{SourceHanSerifSC-SemiBold.otf}
\newcommand{\hei}{\CJKfamily{黑体}}
\newcommand{\song}{\CJKfamily{宋体}}
\newcommand{\bfhei}{\CJKfamily{粗黑体}}
\newcommand{\bfsong}{\CJKfamily{粗宋体}}
\newcommand{\yihao}{\fontsize{26pt}{26pt}\selectfont}       % 一号, 单倍行距
\newcommand{\xiaoyi}{\fontsize{24pt}{24pt}\selectfont}      % 小一, 单倍行距
\newcommand{\erhao}{\fontsize{22pt}{1.25\baselineskip}\selectfont}       % 二号, 1.25倍行距
\newcommand{\xiaoer}{\fontsize{18pt}{18pt}\selectfont}      % 小二, 单倍行距
\newcommand{\sanhao}{\fontsize{16pt}{16pt}\selectfont}      % 三号, 单倍行距
\newcommand{\xiaosan}{\fontsize{15pt}{15pt}\selectfont}     % 小三, 单倍行距
\newcommand{\sihao}{\fontsize{14pt}{14pt}\selectfont}       % 四号, 单倍行距
\newcommand{\xiaosi}{\fontsize{12pt}{12pt}\selectfont}      % 小四, 单倍行距
\newcommand{\wuhao}{\fontsize{10.5pt}{10.5pt}\selectfont}   % 五号, 单倍行距
\newcommand{\xiaowu}{\fontsize{9pt}{9pt}\selectfont}        % 小五, 单倍行距

%\CJKtilde  % 重新定义了波浪符~的意义
% JUST DON'T USE CJK
% 使用 ctexbook 之后已无必要
\newcommand\prechaptername{第}
\newcommand\postchaptername{章}

\punctstyle{hangmobanjiao}             % 调整中文字符的表示,行内占一个字符宽度,行尾占半个字符宽度

% 调整罗列环境的布局
\setitemize{leftmargin=3em,itemsep=0em,partopsep=0em,parsep=0em,topsep=-0em}
\setenumerate{leftmargin=3em,itemsep=0em,partopsep=0em,parsep=0em,topsep=0em}

% 避免宏包 hyperref 和 arydshln 不兼容带来的目录链接失效的问题。
\def\temp{\relax}
\let\temp\addcontentsline
\gdef\addcontentsline{\phantomsection\temp}

% 自定义项目列表标签及格式 \begin{publist} 列表项 \end{publist}
\newcounter{pubctr} %自定义新计数器
\newenvironment{publist}{%%%%%定义新环境
\begin{list}{[\arabic{pubctr}]} %%标签格式
    {
     \usecounter{pubctr}
     \setlength{\leftmargin}{2.5em}   % 左边界 \leftmargin =\itemindent + \labelwidth + \labelsep
     \setlength{\itemindent}{0em}     % 标号缩进量
     \setlength{\labelsep}{1em}       % 标号和列表项之间的距离,默认0.5em
     \setlength{\rightmargin}{0em}    % 右边界
     \setlength{\topsep}{0ex}         % 列表到上下文的垂直距离
     \setlength{\parsep}{0ex}         % 段落间距
     \setlength{\itemsep}{0ex}        % 标签间距
     \setlength{\listparindent}{0pt}  % 段落缩进量
    }}
{\end{list}}

\makeatletter
\renewcommand\normalsize{
  \@setfontsize\normalsize{12pt}{12pt} % 小四对应 12 pt
  \setlength\abovedisplayskip{4pt}
  \setlength\abovedisplayshortskip{4pt}
  \setlength\belowdisplayskip{\abovedisplayskip}
  \setlength\belowdisplayshortskip{\abovedisplayshortskip}
  \let\@listi\@listI}
\def\defaultfont{\renewcommand{\baselinestretch}{1.63}\normalsize\selectfont} % 设置行距

\renewcommand{\CJKglue}{\hskip -0.1 pt plus 0.08\baselineskip} % 控制字间距,使每行 34 个汉字
\makeatother

%%%%%%%%%%%%% Contents %%%%%%%%%%%%%%%%%
\renewcommand{\contentsname}{目\qquad 录}
\setcounter{tocdepth}{2} % 控制目录深度
% 使用 ctexbook 之后已无必要
\titlecontents{chapter}[2em]{\vspace{.5\baselineskip}\xiaosan\song}
             {\prechaptername\CJKnumber{\thecontentslabel}\postchaptername\qquad}{}
             {\hspace{.5em}\titlerule*[10pt]{$\cdot$}\sihao\contentspage}
\titlecontents{chapter}[2em]{\vspace{.25\baselineskip}\xiaosi\song}
             {\thecontentslabel\!\!\qquad}{}
             {\titlerule*[5pt]{$\cdot$}\xiaosi\contentspage}
\titlecontents{section}[3em]{\vspace{.25\baselineskip}\xiaosi\song}
             {\thecontentslabel\quad}{}
             {\titlerule*[5pt]{$\cdot$}\xiaosi\contentspage}
\titlecontents{subsection}[4em]{\vspace{.25\baselineskip}\xiaosi\song}
             {\thecontentslabel\quad}{}
             {\titlerule*[5pt]{$\cdot$}\xiaosi\contentspage}

%%%%%%%%%% Chapter and Section %%%%%%%%%%%%%
\setcounter{secnumdepth}{4}
\setlength{\parindent}{2em}
\renewcommand{\chaptername}{\prechaptername\CJKnumber{\thechapter}\postchaptername}
\renewcommand\thesection{\arabic {section}.}
\renewcommand\thesubsection{\arabic {section}.\arabic {subsection}.}
\titleformat{\chapter}{\centering\xiaosan\hei}{\chaptername}{2em}{}
\titlespacing{\chapter}{0pt}{0.1\baselineskip}{0.8\baselineskip}
\titleformat{\section}{\sihao\hei}{\thesection}{1em}{}
\titlespacing{\section}{0pt}{0.15\baselineskip}{0.25\baselineskip}
\titleformat{\subsection}{\sihao\hei}{\thesubsection}{1em}{}
\titlespacing{\subsection}{0pt}{0.1\baselineskip}{0.3\baselineskip}
\titleformat{\subsubsection}{\sihao\hei}{\thesubsubsection}{1em}{}
\titlespacing{\subsubsection}{0pt}{0.05\baselineskip}{0.1\baselineskip}

\newcommand{\trtitle}[1]{\vspace{12pt}\begin{center}{\xiaoer\hei\bf #1}\end{center}}
\newcommand{\trabs}[1]{\vspace{6pt}{\begin{center}{\xiaosan\hei #1}\end{center}}}
\newcommand{\trchapter}[1]{\vspace{6pt}{\raggedright\xiaosan\hei #1}}%%\newline}
\newcommand{\trsection}[1]{\vspace{4pt}{\raggedright\sihao\hei #1}}%%%\newline}
\newcommand{\trsubsection}[1]{\vspace{2pt}{\raggedright\sihao\hei\it #1}}%%%\newline}

%%%%%%%%%% Table, Figure and Equation %%%%%%%%%%%%%%%%%
\renewcommand{\tablename}{表}                                     % 插表题头
\renewcommand{\figurename}{图}                                    % 插图题头
%\renewcommand{\thefigure}{\arabic{chapter}-\arabic{figure}}       % 使图编号为 7-1 的格式 %\protect{~}
\renewcommand{\thesubfigure}{\alph{subfigure})}                   % 使子图编号为 a) 的格式
\renewcommand{\thesubtable}{(\alph{subtable})}                    % 使子表编号为 (a) 的格式
%\renewcommand{\thetable}{\arabic{chapter}-\arabic{table}}         % 使表编号为 7-1 的格式
\renewcommand{\theequation}{\arabic{section}-\arabic{equation}}   % 使公式编号为 7-1 的格式
\newcommand{\ud}{\mathrm{d}}

%%%%%% 定制浮动图形和表格标题样式 %%%%%%
\makeatletter
\long\def\@makecaption#1#2{
   \vskip\abovecaptionskip
   \sbox\@tempboxa{\centering\wuhao\song{#1\quad #2} }
   \ifdim \wd\@tempboxa >\hsize
%     \centering\wuhao\song{#1\qquad #2} \par
     \wuhao\song{#1\quad #2} \par
   \else
     \global \@minipagefalse
     \hb@xt@\hsize{\hfil\box\@tempboxa\hfil}
   \fi
   \vskip\belowcaptionskip\vspace{8pt}}
\makeatother
\captiondelim{~~~~} %用来控制longtable表头分隔符

%%%%%%%%%% Theorem Environment %%%%%%%%%%%%%%%%%
\theoremstyle{plain}
\theorembodyfont{\song\rmfamily}
\theoremheaderfont{\hei\rmfamily}
\newtheorem{theorem}{定理~}[chapter]
\newtheorem{lemma}{引理~}[chapter]
\newtheorem{axiom}{公理~}[chapter]
\newtheorem{proposition}{命题~}[chapter]
\newtheorem{prop}{性质~}[chapter]
\newtheorem{corollary}{推论~}[chapter]
\newtheorem{definition}{定义~}[chapter]
\newtheorem{conjecture}{猜想~}[chapter]
\newtheorem{example}{例~}[chapter]
\newtheorem{remark}{注~}[chapter]
%\newtheorem{algorithm}{算法~}[chapter]
\newenvironment{proof}{\noindent{\hei 证明:}}{\hfill $ \square $ \vskip 4mm}
\theoremsymbol{$\square$}

%%%%%%%%%% Page: number, header and footer  %%%%%%%%%%%%%%%%%

%\frontmatter 或 \pagenumbering{roman}
%\mainmatter 或 \pagenumbering{arabic}
\makeatletter
\renewcommand\frontmatter{\clearpage
  \@mainmatterfalse
  }
\makeatother

%%%%%%%%%%% Code: Listings from MCM Template %%%%%%%%%%%%

\definecolor{grey}{rgb}{0.8,0.8,0.8}
\definecolor{darkgreen}{rgb}{0,0.3,0}
\definecolor{darkblue}{rgb}{0,0,0.3}
\def\lstbasicfont{\fontfamily{pcr}\selectfont\footnotesize}
\lstset{%
% indexing
   % numbers=left,
   % numberstyle=\small,%
% character display
    showstringspaces=false,
    showspaces=false,%
    tabsize=4,%
% style
    frame=lines,%
    basicstyle={\footnotesize\lstbasicfont},%
    keywordstyle=\color{darkblue}\bfseries,%
    identifierstyle=,%
    commentstyle=\color{darkgreen},%\itshape,%
    stringstyle=\color{black}%
}
\lstloadlanguages{C,C++,Java,Matlab,Mathematica,Python}

%%%%%%%%%%%% References %%%%%%%%%%%%%%%%%

\makeatletter
\renewenvironment{thebibliography}[1]{
   \wuhao
   \list{\@biblabel{\@arabic\c@enumiv}}
        {\renewcommand{\makelabel}[1]{##1\hfill}
         \settowidth\labelwidth{0 cm}
         \setlength{\labelsep}{0pt}
         \setlength{\itemindent}{0pt}
         \setlength{\leftmargin}{\labelwidth+\labelsep}
         \addtolength{\itemsep}{-0.7em}
         \usecounter{enumiv}
         \let\p@enumiv\@empty
         \renewcommand\theenumiv{\@arabic\c@enumiv}}
    \sloppy\frenchspacing
    \clubpenalty4000
    \@clubpenalty \clubpenalty
    \widowpenalty4000
    \interlinepenalty4000
    \sfcode`\.\@m}
   {\def\@noitemerr
     {\@latex@warning{Empty 'thebibliography' environment}}
    \endlist\frenchspacing}
\makeatother

\addtolength{\bibsep}{-0.5em}     % 缩小参考文献间的垂直间距
\setlength{\bibhang}{2em}         % 每个条目自第二行起缩进的距离

% 参考文献引用作为上标出现
\newcommand{\citeup}[1]{\textsuperscript{\cite{#1}}}

%%%%%%%%%%%% Cover %%%%%%%%%%%%%%%%%
% 封面、摘要、版权、致谢格式定义
\makeatletter
\def\title#1{\def\@title{#1}}\def\@title{}
\def\cdegree#1{\def\@cdegree{#1}}\def\@cdegree{}
\def\caffil#1{\def\@caffil{#1}}\def\@caffil{}
\def\csubject#1{\def\@csubject{#1}}\def\@csubject{}
\def\cgrade#1{\def\@cgrade{#1}}\def\@cgrade{}
\def\author#1{\def\@author{#1}}\def\@author{}
\def\email#1{\def\@email{#1}}\def\@email{}
\def\cnumber#1{\def\@cnumber{#1}}\def\@cnumber{}
\def\csupervisor#1{\def\@csupervisor{#1}}\def\@csupervisor{}
\def\crank#1{\def\@crank{#1}}\def\@crank{}
\def\cdate#1{\def\@cdate{#1}}\def\@cdate{}
\long\def\cindependent#1{\long\def\@cindependent{#1}}\long\def\@cindependent{}
\long\def\cabstract#1{\long\def\@cabstract{#1}}\long\def\@cabstract{}
\long\def\eabstract#1{\long\def\@eabstract{#1}}\long\def\@eabstract{}
\def\ckeywords#1{\def\@ckeywords{#1}}\def\@ckeywords{}
\def\ekeywords#1{\def\@ekeywords{#1}}\def\@ekeywords{}
\def\cheading#1{\def\@cheading{#1}}\def\@cheading{}


\pagestyle{fancy}
\fancyhf{}
\fancyhead[C]{\song\wuhao \@cheading}  % 页眉显示天津大学 20XX 届本科生毕业论文
\fancyfoot[C]{\song\xiaowu ~\thepage~}
\newlength{\@title@width}

%%%%%%%%%%%%%%%%%%%     Cover     %%%%%%%%%%%%%%%%%%%%%%%
\def\makecover{
	\newpage
	\null
	\vskip .375in
	\begin{center}
		{\Large \bf \@title \par}
		% additional two empty lines at the end of the title
		\vspace*{24pt}
		{
			\large
			\lineskip .5em
			%\resizebox{1\textwidth}{!}{
			\begin{tabular}[t]{cp{1\textwidth}}
				\@author\\
				\@email
			\end{tabular}%}
			\par
		}
		% additional small space at the end of the author name
		\vskip .5em
		% additional empty line at the end of the title block
		\vspace*{12pt}
	\end{center}
}

\def\abstract
{%
	\centerline{\bfsong\xiaosan 摘\qquad 要}%
	\vspace*{12pt}%
}

\def\endabstract
{
	% additional empty line at the end of the abstract
	\vspace*{12pt}
}

%%%%%%%%%%%%%%%%%%%   Independent  %%%%%%%%%%%%%%%%%%%%%%%
\def\makeindependent{
	\titleformat{\chapter}{\centering\erhao\bfsong}{\chaptername}{2em}{}
	
	\clearpage
	\markboth{独创性声明}{独创性声明}
	\pdfbookmark[0]{独创性声明}{cindependent}
	\chapter*{独创性声明}
	\song\sanhao\defaultfont
	\@cindependent
	\vspace{\baselineskip}
	\thispagestyle{empty}
}
%%%%%%%%%%%%%%%%%%%    Abstract    %%%%%%%%%%%%%%%%%%%%%%%
\def\makeabstract{
	\titleformat{\chapter}{\centering\erhao\bfsong}{\chaptername}{2em}{} %"摘要"二号宋体加粗居中

%%%%% Chinese Abstract %%%%%
%	\clearpage
	\markboth{摘~要}{摘~要}
	\pdfbookmark[0]{摘~~要}{cabstract}
	\chapter*{摘\qquad 要}
	
	\song\defaultfont
	\@cabstract
	\vspace{\baselineskip}

	\hangafter=1\hangindent=55pt
	\noindent
	{\bfsong\sihao{关键词:}} \@ckeywords %%%%%跟原文空一行小四12pt
}

%%%%%%%%%%%%%%%%%%%    Contents    %%%%%%%%%%%%%%%%%%%%%%%
\def\makecontents{

	\mainmatter\defaultfont\sloppy\raggedbottom
	\makeatletter
	\fancypagestyle{plain}{                              % 设置开章页眉页脚风格
    	\fancyhf{}
    	\fancyhead[C]{\song\wuhao \@cheading}            % 首页页眉格式
    	\fancyfoot[C]{\song\xiaowu ~\thepage~}           % 首页页脚格式
    	\renewcommand{\headrulewidth}{0.5pt}
    	\renewcommand{\footrulewidth}{0pt}
	}
}

%%%%%%%%%%%%%%%%%%%   References   %%%%%%%%%%%%%%%%%%%%%%%
\def\makereferences{
%	\titleformat{\chapter}{\centering\xiaosan\hei}{\chaptername}{2em}{} % "参考文献"小三黑体顶格
	\setlength{\baselineskip}{17pt}
	\setlength{\parskip}{3pt}
	\section*{\bibname}
	\defaultfont
	\bibliographystyle{references/ref.bst}
	\bibliography{references/ref}

	\phantomsection
	\markboth{参考文献}{参考文献}
%	\addcontentsline{toc}{chapter}{参考文献}       % 参考文献加入到中文目录
	%\nocite{*}                                     % 若将此命令屏蔽掉,则未引用的文献不会出现在文后的参考文献中
}

\makeatother

